%%================================================
%% Filename: abstract.tex
%% Encoding: UTF-8
%% Author: Yuan Xiaoshuai - yxshuai@gmail.com
%% Created: 2012-04-24 00:21
%% Last modified: 2019-11-01 09:26
%%================================================
\begin{cabstract}
在高速信息化的时代背景下,带宽的数量级不断提升,数据连接速度愈发迅捷,网络技术也日新月异,这些变革为用户带来了前所未有的体验。
然而,与之相伴的是网络攻击形式也愈发多样和复杂,为新形势下的网络安全带来了严峻挑战。
当网络信息系统遭受攻击时,如何迅速而有效地应对,成为摆在我们面前的一大难题。
这包括但不限于如何精准地检测到攻击行为、如何确定攻击者的真实身份,以及如何及时采取相应措施进行防御。
这些问题迫切需要得到有效的解决。
基于当前的技术调研与分析,本文在网络攻击检测和网络攻击溯源两个关键研究方向上进行了深入的研究。
分别作出了以下工作:\par
(1) 鉴于传统概率包标记法在网络溯源领域存在的溯源准确率低、溯源速度慢等弊端,本文进行了针对性的改进,提出了一种新颖的基于路由器接口分组概率包标记的方案。
该方案通过三个精心设计的阶段——路由器标记、数据包收集与路径重构,实现了高效的网络溯源。
在路由器标记阶段,我们充分利用数据包传输过程中经过的路由器信息,对其进行标记写入到数据包中,从而有效收集数据包在网络中的传输路径。
随后,在数据包收集阶段,我们从这些标记数据包中提取这些标记信息,并启动相应的路径重构程序。
最后,在路径重构阶段,我们巧妙运用异或运算的特性,从收集的标记信息中解析出具体的路由器接口号序列路径,再通过递归的方式,我们便可踪到数据包的发送源。
实验评估证明,与传统的PPM方案相比,本方案在溯源速度和准确性上均表现出色,为网络溯源任务提供了强有力的支持。\par

(2) 针对当前攻击检测模型不能充分利用流量数据中的空间特征与时序特征,本文提出了一种基于残差网络与双向门控循环单元多模态特征融合的模型。
该模型结合了残差网络和双向门控循环单元的优势,有效挖掘流量数据的时空特征,并通过综合分析产生更为精准的决策。
在UNSW-NB15数据集上的实验对比显示,本模型在准确率方面相较传统机器学习模型以及ResNet、BiGRU等模型具有显著优势,为攻击检测任务提供了更加可靠的解决方案。\par

(3) 针对传统攻击检测模型难以适应动态变化的网络攻击环境,本文创新性地将所提出的RNB-MF模型与iCaRL策略相结合,设计出了一套高效的增量学习方案。
这一方案不仅继承了RNB-MF模型在特征提取和攻击检测方面的优势,还借助iCaRL策略实现了对模型知识的有效保留和更新。
通过不断地学习新的攻击模式,同时保留对已知攻击的检测能力,该方案显著提高了模型在复杂多变环境下的适应性。
此外,为了进一步提升方案的性能,本文还引入了基于遗传算法的记忆集优化方法。
通过对记忆集进行智能优化,该方法有效提高了模型在增量学习过程中的效率和准确性,进一步增强了方案的实用性。
最后经过实验验证,该方案能够使模型不断适应日益变化的网络攻击环境,确保网络安全的持续性和稳定性。\par
\end{cabstract}

\ckeywords{异常检测, 网络溯源, 深度学习, 数据包标记法, 残差网络, 增量学习, 门控循环单元}

\begin{eabstract} 

In the backdrop of rapid information age advancements, the magnitude of bandwidth continues to soar, data connection speeds are increasingly swift, and network technology evolves by the day.
These changes have ushered in unprecedented user experiences. However, accompanying these advancements, the diversity and complexity of cyberattacks have also increased, posing severe challenges to cybersecurity in this new era. Rapid and effective response strategies to cyberattacks are crucial challenges we face, encompassing precise attack detection, accurate identification of attackers, and timely defensive measures. 
These issues urgently need effective solutions. 
Based on current technical research and analysis, this paper delves deeply into two key research directions: cyberattack detection and cyberattack provenance. 
The contributions are as follows:


  (1) Given the low accuracy and slow speed of traditional probability packet marking methods in cyber provenance, this paper introduces a novel scheme based on probabilistic packet marking at router interfaces. 
  This scheme, through three meticulously designed phases—router marking, packet collection, and path reconstruction—achieves efficient cyber provenance.
  During the router marking phase, we leverage the information of routers traversed during packet transmission to mark packets, effectively collecting the packets’ transmission paths within the network. 
  Then, in the packet collection phase, we extract these markings and initiate the path reconstruction program. Finally, in the path reconstruction phase, we cleverly use the properties of XOR operations to decode the sequence of router interface numbers from the collected markings. 
  Through recursion, we can trace the packet’s origin. Experimental evaluations demonstrate that our scheme significantly outperforms traditional PPM methods in speed and accuracy, providing strong support for cyber provenance tasks.\par
  
  
  (2) Addressing the current attack detection models’ inability to fully utilize spatial and temporal features in traffic data, this paper proposes a multimodal feature fusion model based on residual networks and bidirectional gated recurrent units. 
  This model combines the strengths of residual networks and bidirectional gated recurrent units to effectively mine spatiotemporal features in traffic data, resulting in more accurate decision-making. 
  Experimental comparisons on the UNSW-NB15 dataset show that our model significantly surpasses traditional machine learning models and other models such as ResNet and BiGRU in accuracy, offering a more reliable solution for attack detection tasks.\par
  
  
  (3) To tackle the challenge that traditional attack detection models face in adapting to the ever-changing landscape of network attacks, this paper combines the proposed RNB-MF model with the iCaRL strategy, designing an efficient incremental learning scheme. 
  This scheme incorporates a memory set optimization method based on genetic algorithms, further enhancing the performance of the approach. 
  Experiments on the UNSW-NB15 and NSL-KDD datasets demonstrate that our scheme supports incremental learning and knowledge retention, enabling the model to continuously adapt to evolving network attacks, thereby ensuring the sustainability and stability of cybersecurity.\par

\end{eabstract}

\ekeywords{anomaly detecion, network traffic, network traceback,
packet marking traceback, deep learning, incremental learning}
