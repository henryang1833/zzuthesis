%%================================================
%% Filename: abstract.tex
%% Encoding: UTF-8
%% Author: Yuan Xiaoshuai - yxshuai@gmail.com
%% Created: 2012-04-24 00:21
%% Last modified: 2019-11-01 09:26
%%================================================
\begin{cabstract}
在高速信息化的时代浪潮中,带宽的数量级不断攀升,数据传输速度愈发迅猛,网络技术更是日新月异,为用户带来了前所未有的便捷体验。
然而,与此同时,网络攻击的形式也愈发多样化和复杂化,给新形势下的网络安全带来了前所未有的严峻挑战。
异常流量检测技术,作为能够有效检测未知网络攻击的重要手段,已然成为网络入侵检测领域的核心研究方向。
然而,现有的检测模型往往仅聚焦于流量数据的空间维度特征,这在面对日益复杂且多维度的流量数据时显得力不从心,极大地削弱了模型的检测效能。
此外,当前的网络攻击检测技术多依赖于静态数据集,这在瞬息万变的网络环境中显得尤为捉襟见肘,实际应用中面临着诸多未知的挑战。
更为严重的是,当模型在实际运行时,可能会遭遇DDoS等恶意攻击,导致模型负担过重,甚至面临崩溃的风险。
这些问题迫切需要得到有效的解决。
本文对当前的技术进行了调研与分析,分别作出了以下工作:\par

1)~针对现有检测模型往往仅关注流量数据的空间特征,而忽视多维度特征的综合考量,导致异常检测效果存在局限性的问题,本文提出了一种基于残差网络与双向门控循环单元多模态特征融合模型。
该模型结合了残差网络和双向门控循环单元的优势,有效挖掘流量数据的时空特征,并通过综合分析产生更为精准的决策。
在UNSW-NB15数据集上的实验结果显示,该模型准确率高达99.46\%,并在各类攻击样本的精度和召回率指标上也取得了卓越表现,为攻击检测任务提供了更加可靠的解决方案。\par

2)~针对传统攻击检测模型难以适应动态变化的网络攻击环境问题,本文将所提出的融合模型与iCaRL策略相结合,设计出了一套高效的增量学习方案。
这一方案不仅继承了融合模型在特征提取和攻击检测方面的优势,还借助iCaRL策略实现了对模型知识的有效保留和更新。
通过不断学习新的攻击模式,同时保留对已知攻击的检测能力,该方案显著提高了模型在复杂多变环境下的适应性。
此外,为了进一步提升方案的性能,本文还引入了基于遗传算法的记忆集优化方法。
通过对记忆集进行智能优化,该方法有效提高了模型在增量学习过程中的效率和准确性,进一步增强了方案的实用性。
最后通过利用UNSW-NB15和NSL-KDD数据集分别作为新旧任务数据集对模型进行实验验证,该模型在新旧任务混合数据集上准确率高达78.1\%,在旧任务数据集上准确率达75.6\%。
实验结果表明,该方案能够使模型不断适应日益变化的网络攻击环境,确保网络安全的持续性和稳定性。\par

3)~针对检测模型在实际运行过程中可能面临DDoS攻击威胁的问题,本文提出了基于路由器接口号成组标记溯源方法,之后在此基础上本文设计并实现了基于溯源的模型防护方法。
为验证该溯源方法的效果,本文利用CAIDA IPv4 Prefix-Probing Traceroute数据集对其进行了溯源速度测试。
实验结果表明,在使用“组数”为4的条件下,仅需收集65个数据包,该方法便能成功追踪到距离接收结点30跳路由器的数据发送结点。
紧接着,我们进一步利用NS-3模拟器构造了一个网络仿真环境,以测试该方法在面对实际DDoS攻击时的溯源准确率。
实验结果显示,随着攻击节点数量不断增加,直至达到100个的情况下,溯源准确率依然能够稳定在90\%以上。
最后,为验证防护方法在模型实际运行时的表现效果,本文通过VMWare WorkStation构建了一个高度仿真的网络攻击环境,对防护方法进行了实验验证。
实验结果表明,该防护方法能够有效定位攻击者的最近邻路由器,并依托该路由器实现对攻击流量的过滤,从而保障检测模型的稳定运行和服务器的安全无虞。
\end{cabstract}

\ckeywords{异常检测, 网络溯源, 深度学习, 数据包标记法, 残差网络, 增量学习, 门控循环单元}

\begin{eabstract} 

In the era of high-speed information technology, the magnitude of bandwidth continues to climb, data connection speeds are becoming increasingly rapid, and network technology is constantly evolving, bringing unprecedented convenient experiences to users.
However, at the same time, the forms of network attacks are becoming more diverse and complex, posing unprecedented challenges to network security in the new situation.
As an important means of effectively detecting unknown network attacks, abnormal traffic detection technology has become a core research direction in the field of network intrusion detection.
However, existing detection models often focus only on single-dimensional features of traffic data, which makes them unable to cope with increasingly complex and multi-dimensional traffic data, greatly reducing the detection effectiveness of the models.
In addition, current network attack detection technologies often rely on static datasets, which is particularly limiting in the rapidly changing network environment and faces many unknown challenges in practical applications.
Even more seriously, when models are actually running, they may encounter malicious attacks such as DDoS, leading to excessive load on the model and even the risk of collapse.
These issues urgently need to be effectively addressed.

This article conducted a survey and analysis of current technologies and made the following contributions:\par

1)~In response to the current attack detection models' inability to fully utilize the spatial and temporal features in traffic data, this article proposes a model based on the fusion of multi-modal features using residual networks and bidirectional gated recurrent units.)
This model combines the advantages of residual networks and bidirectional gated recurrent units to effectively mine the spatiotemporal features of traffic data and produce more accurate decisions through comprehensive analysis.
Experimental results on the UNSW-NB15 dataset show that the model achieves an accuracy rate of up to 99.46\% and excels in various attack sample precision and recall metrics, providing a more reliable solution for attack detection tasks.\par

2)~Addressing the difficulty of traditional attack detection models in adapting to dynamically changing network attack environments, this article innovatively combines the proposed fusion model with the iCaRL strategy to design an efficient incremental learning scheme.
This scheme not only inherits the advantages of the fusion model in feature extraction and attack detection but also utilizes the iCaRL strategy to effectively retain and update model knowledge.
By continuously learning new attack patterns while retaining the ability to detect known attacks, this scheme significantly improves the model's adaptability in complex and changing environments.
Furthermore, to further enhance the performance of the scheme, this article introduces a memory set optimization method based on genetic algorithms.
By intelligently optimizing the memory set, this method effectively improves the efficiency and accuracy of the model during incremental learning, further enhancing the practicality of the scheme.
Finally, experiments were conducted using the UNSW-NB15 and NSL-KDD datasets as old and new task datasets, respectively, to validate the model.
The model achieved an accuracy rate of up to 78.1\% on the mixed dataset of old and new tasks and 75.6\% on the old task dataset.
The experimental results demonstrate that this scheme enables the model to continuously adapt to changing network attack environments, ensuring the continuity and stability of network security.\par

3)~In response to the threat of DDoS attacks that detection models may face during actual operation, this article proposes a traceback method based on grouped marking of router interface numbers.
Based on this, a model protection method based on traceback is designed and implemented.
To verify the effectiveness of the traceback method, this article used the CAIDA IPv4 Prefix-Probing Traceroute dataset to test its traceback speed.
The experimental results show that, under the condition of using a "group number" of 4, the method can successfully trace the data sending node of the router 30 hops away from the receiving node by collecting only 65 packets.
Subsequently, we further utilized the NS-3 simulator to construct a network simulation environment to test the traceback accuracy of this method in actual DDoS attacks.
The experimental results show that as the number of attack nodes continues to increase, even reaching 100, the traceback accuracy can still stabilize above 90\%.
Finally, to verify the performance of the protection method during actual model operation, this article constructed a highly simulated network attack environment through VMWare WorkStation to conduct experimental validation of the protection method.
The experimental results demonstrate that the protection method can effectively locate the attacker's nearest neighbor router and rely on that router to filter attack traffic, thus ensuring the stable operation of the detection model and the safety of the server.
\end{eabstract}

\ekeywords{anomaly detecion, network traffic, network traceback,
packet marking traceback, deep learning, incremental learning}
