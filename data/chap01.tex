%%================================================
%% Filename: chap01.tex
%% Encoding: UTF-8
%% Author: Yuan Xiaoshuai - yxshuai@gmail.com
%% Created: 2012-04-27 15:05
%% Last modified: 2016-08-28 21:07
%%================================================
\chapter{绪论}
\label{cha:overview}

\section{研究背景及意义}

近年来随着普及和使用率持续增长,互联网已然成为全球信息交流、经济发展和社会融合的关键基础
设施,在现代社会中有着不可或缺的地位。中国互联网络信息中心(China Internet 
Network Information Center,CNNIC)第 48 次《中国互联网络发展状况统计报告》显
示\cite{cnnic2021}, 截至 2021 年 6 月,我国所使用的 IPv4 地址总数接近 4 亿,IPv6 地址总数
超过了 6万。根据ITU(International Telecommunication Union)的《事实与数据2023报告》显示\cite{itu2023},截止2022年12月,全球共有网民数49.5亿;
截至2023年,全球约有67\%的人口,即大约54亿人正在使用互联网。图~\ref{fig:individula_using_internet}~显示了近几年全球网民数量及普及率的变化。
这些数据表明当前网络规模在急剧扩大,网络多样性也在逐渐增加。
\begin{figure}[htbp]
    \centering
    \includegraphics[width = 0.8\textwidth]{individula_using_internet.png}
    \caption{全球网民数量及普及率变化\cite{fig-itu2023}}
    \label{fig:individula_using_internet}
\end{figure} 


尽管互联网规模的扩大为社会经济和科技进步提供了强大的推动力,并为全人类带来了前所未有的便利,但它却是一把双刃剑,伴
随而来的是一系列隐患和威胁,对人民的生活和安全、甚至是国家的稳定都构成了严重的威胁。
2021年,未知攻击者窃取GoDaddy源代码并在其服务器上安装恶意软件,影响了120万Managed WordPress客户的个人信息。
2022年12月,PayPal遭到凭证填充攻击,攻击者通过尝试使用从其他网站数据泄露中获得的用户名和密码对,成功访问了34,942个账户。
2023年5月末,由Clop勒索软件团伙发起了MOVEit攻击。这次攻击利用了Progress MOVEit文件传输工具的关键漏洞,影响了
2,667个组织和近8400万个人,MGM也因此事件估计损失了1亿美元​。此事件被认为是2023
年范围最广的攻击之一,也是近年来最大的数据盗窃事件之一。图~\ref{fig:daily_web_application_attacks}~展示了Akamai关于2022年
与2023年网络攻击每日统计数量。
\begin{figure}[htbp]
  \centering
  \includegraphics[width = 0.8\textwidth]{daily_web_application_attacks.png}
  \caption{2022年与2023年网络攻击每日统计数量\cite{akamai}}
  \label{fig:daily_web_application_attacks}
\end{figure}
这些数据表明,随着互联网的广泛普及,网络攻击者日益有组织、专业化,新型网络攻击层出不穷,其特性日益高维、复杂,难以识别。\par

面对当前日益严峻的网络安全形势,有效识别和应对不断涌现且特征维度持续攀升的新型网络攻击,对于确保网络安全具有不可估量的重要意义。
虽然现有的网络攻击检测技术相比过去已有显著进步,但面对复杂多变的网络环境,仍存在诸多挑战。
首先,海量流量数据日益高维化、复杂化,其多维特征使得单一的检测维度难以全面捕捉。
因此,网络攻击检测模型的准确率受到限制。
若能从多个维度综合分析流量数据,必将进一步提升检测的准确度。
其次,当前的异常流量检测模型大多依赖于静态数据集,难以适应网络环境的动态变化。
特别是随着新型网络攻击的不断涌现,模型的检测性能往往大幅下降。
因此,设计一个既能保持高检测准确率又能灵活应对新型攻击的模型成为迫切需求。
最后,现有的异常流量检测模型主要停留在被动检测层面,缺乏主动处理能力和源头解决方案。
例如,在面对DDoS攻击时,若模型仅依赖源IP地址过滤异常流量,一旦攻击者频繁更换僵尸节点的IP地址,模型将不堪重负,甚至崩溃。
因此,设计一种有效的攻击溯源方法,成为当前亟待解决的问题。
当模型检测到攻击后,利用该方法成功定位到攻击流量的源头,从而在源头上解决异常流量问题,以保障检测模型的稳定运行和服务器的安全无虞。

\section{国内外研究现状}

\subsection{网络攻击检测研究现状}
异常流量检测的研究主要可以分为基于传统机器学习方法和基于深度学习的方法。\par
1)~基于机器学习的方法\par
在18世纪50年代,机器学习技术还处于其早期探索阶段,标志性事件之一是1952年亚瑟·萨缪尔开发的能够在IBM 701上运行的西洋跳棋程序,这是机器学习应用的早期例证。
进入60年代,随着最近邻算法(Nearest Neighbor Algorithm)\cite{taunk2019brief}的提出,模式识别和机器学习技术开始取得初步发展。
随后的几十年见证了机器学习技术的快速成熟,诸如决策树\cite{charbuty2021classification}、支持向量机(SVM)\cite{zhang2021support}、随机森林\cite{athey2019generalized}等算法相继被提出。
这些算法的发展不仅推动了机器学习领域的进步,也为后来的攻击检测技术打下了坚实的基础。
特别是,支持向量机(SVM)和随机森林等算法开始被集成到入侵检测系统中,标志着基于机器学习方法的网络攻击检测技术的蓬勃发展\cite{amaran2021intrusion}。
这些基于机器学习的攻击检测技术,具体可分为有监督学习和无监督学习。



有监督的机器学习攻击检测技术主要依赖于预先标记的数据集来训练模型,使其能够识别和区分正常行为与恶意行为。
这种技术通常会将数据集分为训练集和测试集,其中训练集用于训练模型,而测试集则用于评估模型的性能。
有监督学习的核心在于获取大量已明确标记为正常行为或攻击行为的样本数据。
这些数据样本对于机器学习算法至关重要,它们能够帮助算法“学习”并准确识别各种攻击行为的特征模式。
在有监督的机器学习攻击检测领域,主要可以分为以下几种类型:基于异常的检测、基于签名的检测、以及基于行为的检测。
每种类型都有其独特的方法和应用场景。


基于异常的检测技术通常依赖于机器学习模型来识别与已知正常行为模式显著偏离的行为。
早期的工作可以追溯到1990年代,其中Lee和Stolfo等人\cite{lee1998dataMining}在1998年的研究中对异常检测技术做出了重要贡献,提出了使用数据挖掘技术来识别网络入侵的方法。
之后Schonlau等人在2001年提出的MASQUE模型,旨在通过监视用户行为来检测异常,进而识别潜在的攻击活动。

基于签名的检测技术则是基于已知攻击的特定模式或签名来识别攻击。
这种方法的关键在于建立一个广泛的攻击特征库。
Kumar和Spafford\cite{kumar1994adep}在1994年的研究中,通过他们的工作ADEPTS,为基于签名的检测技术做出了显著贡献,发展了一种基于异常检测和签名匹配的混合方法来提高检测的准确性。

基于行为的检测技术关注于分析系统或网络中的行为模式,以识别潜在的恶意活动。
这种方式侧重于学习和理解系统或用户的正常行为模式,以便及时发现并标记任何与这些模式有明显偏离的、具有恶意的行为。
在这个方向上,Anderson等人\cite{anderson1995userBehavior}在1995年提出了一种基于用户行为分析的入侵检测系统,该系统通过学习正常的用户行为模式来识别异常活动。
这种类型的研究在2000年代初期得到了加强,其中Forrest等人\cite{forrest1996selfImmune}在自身免疫系统的概念上的工作为基于行为的检测提供了理论基础,推动了基于行为的检测技术的发展。



无监督的机器学习攻击检测技术不依赖于预先标记的数据集,而是通过分析数据中的模式、异常或者关系来识别可能的恶意行为。
这种技术尤其适用于那些标签数据稀缺或者完全没有标签的情况,允许模型自行发现数据集中的结构。
在无监督学习领域,主要的攻击检测方法包括聚类、异常检测以及密度估计等。

聚类是一种常见的无监督学习技术,通过将数据分成由相似对象组成的多个组来工作。
在攻击检测中,聚类方法可以用来识别出不寻常的数据点集合,这些集合可能代表了恶意活动。
Portnoy等人\cite{portnoy2001clustering}在2001年的研究中通过聚类技术成功识别出异常行为,展示了聚类方法在入侵检测中的有效性。

异常检测技术在无监督学习中也占有一席之地,与有监督学习中的基于异常的检测技术相似,但不依赖于预标记的数据。
Eskin等人\cite{eskin2002anomaly}在2002年的工作中提出了一种基于异常检测的入侵检测系统,该系统能够在没有任何先验知识的情况下识别攻击行为。

密度估计方法,如局部异常因子(LOF)算法,通过评估数据点在其邻域的密度来识别异常点。
此方法特别适合用于检测位于低密度区域的数据点,这些点可能代表着异常或潜在的攻击行为。
Schubert等人\cite{schubert2014local}在2014年对LOF算法进行了改进,以增强其在检测网络入侵方面的性能。

虽然传统机器学习方法在处理一些简单的异常流量检测问题上相对方便,但在应对复杂、非结构化以及高维度流量特征数据时很难有效地学习到深层次的特征。
因此,越来越多的研究人员开始采用深度学习方法,以提高网络异常流量检测的效率和准确性。\par

2)~基于深度学习的方法\par
深度学习是机器学习的一个分支,它通过使用具有多个隐藏层的神经网络来模拟人类大脑的工作方式,从而解决复杂的模式识别问题。
1943年,McCulloch和Pitts提出了最早的神经网络模型。
1986年,Rumelhart, Hinton和Williams重新引入了反向传播算法,使得训练多层神经网络成为可能。
2006年,Hinton和他的学生提出了深度置信网络,这标志着深度学习的复兴。
随后,在2010年代初,随着深度学习技术的发展,研究人员开始探索其在网络安全和攻击检测中的应用。
深度学习在处理和分析大规模复杂数据方面表现出色,因此它成为了提高攻击检测性能的一种有力工具。
基于深度学习的攻击检测技术,根据其学习和推理的方式,可以大致分为生成式方法、判别式方法和混合式方法。

生成式方法旨在学习数据的联合分布$P(X,Y)$,其中$X$是输入数据,$Y$是标签。
这种方法通过模拟数据生成过程来理解和识别正常与异常模式。
在攻击检测中,生成式模型可以用于生成与真实数据分布相似的样本,从而帮助识别潜在的攻击行为。
典型的生成式模型包括深度信念网络(DBNs)和生成对抗网络(GANs)。
2014年,Goodfellow等人提出了生成对抗网络(GANs)\cite{goodfellow2014generative},这是一种强大的生成式模型,能够生成与真实数据几乎无法区分的数据样本。
在攻击检测领域,GANs被用于生成攻击数据,以帮助训练模型更好地识别和适应新的攻击类型。
2015年,An和Cho探索了使用变分自编码器(VAE)进行异常检测的方法\cite{an2015variational}。
该方法通过计算输入数据的重构概率来识别出异常数据,这体现了变分自编码器(VAE)在处理复杂数据集时进行无监督异常检测的巨大潜力。\par

判别式方法专注于学习从输入数据到标签的条件概率分布$P(Y∣X)$。
这类方法通过区分不同类别的数据特征来进行分类或预测。
在攻击检测应用中,判别式模型,如卷积神经网络(CNNs)和循环神经网络(RNNs),因能够有效处理和分析大量的网络数据而被广泛用于识别各种类型的网络攻击。
2015年,Saxe和Berlin展示了如何使用深度CNN来检测恶意软件\cite{saxe2015deep}。
他们通过将二进制文件转换为二维图像,然后使用CNN进行特征学习和分类,有效提高了恶意软件检测的准确率。
2019年,Staudemeyer和Morris\cite{staudemeyer2019applying}在论文中展示了如何利用LSTM网络来检测DDoS攻击。
他们利用RNN的时间序列学习能力,有效识别了网络流量中的异常模式。\par

混合式方法结合了生成式和判别式方法的优点,以提高模型的泛化能力和准确性。
这种方法通常涉及使用生成模型来增强数据或特征,随后通过判别模型进行精确的分类或检测。
混合式方法在攻击检测中特别有用,因为它们可以通过生成式组件来理解和模拟攻击行为的复杂性,同时利用判别式组件的强大分类能力来准确识别攻击。
2018年,Li及其团队提出了MAD-GAN\cite{li2018mad},该方法融合了生成对抗网络和深度学习模型的优势,专门用于检测时间序列数据中的多变量异常。
MAD-GAN的独特之处在于,它利用GAN生成的模拟异常数据来训练深度学习模型,显著提升了模型对新型攻击行为的检测敏感度和准确性。
2019年,Feng及其团队提出了一种结合自编码器和卷积神经网络的方法来检测物联网(IoT)网络中的异常行为\cite{feng2019deep}。
该方法首先利用自编码器学习正常网络流量的特征表示,然后使用CNN对流量进行分类,有效识别了异常模式。

\subsection{网络溯源研究现状}
网络溯源方案可以根据用于收集追踪信息的底层方法进行分类。这些方法在部署策略、存储需求、
信息收集算法等方面可能有所不同。下面给出了几种常见的溯源方法\cite{singh2016}。\par

1)~链路测试法\par
链路测试会对所有上游链路进行递归的分析,直到到达源点为止,以确定传递攻击数据包的链路。
图~\ref{fig:linktest}~展示了此方法的基本原理。
2000年,Burch提出了基于这一原则的第一个IP回溯方案\cite{Burch2000Tracing}。随后是一些基于链接测试的回溯方案
\cite{bhavani2020ip,hasan2023deep,
ThingSlomanDulay2008DDoSDetection}。
链路测试方案有两种:输入调试和控制洪水。
\begin{figure}[htbp]
  \centering
  \includegraphics[width = 0.8\textwidth]{linktest.drawio.png}
  \caption{链路测试法原理}
  \label{fig:linktest}
\end{figure}

% 2)~输入调试法\par
输入调试法的核心思想在于利用路由器对特定分组特征的识别能力来确定攻击路径。
在输入调试法的应用中,受害者能够构建一个攻击数据包的签名,并将其发送至上游路由器。
随后,这些路由器能够依据接收到的攻击包签名,递归地追溯其上游链路,从而精准地识别出攻击者的位置。
这种方法具有显著的优点。首先,它与现有的网络协议和基础设施保持高度一致,这意味着在实施过程中无需对网络架构进行大规模的改动,降低了部署的难度和成本。
其次,输入调试法为增量部署提供了良好的支持,可以根据实际需求逐步扩展应用范围,灵活性较高。
此外,该方法在网络流量上的带宽开销很小,不会给网络带来额外的负担,保证了网络的稳定运行。

然而,输入调试法也存在一些不容忽视的缺点。
首先,它并不适用于DDoS环境,因为在分布式拒绝服务攻击中,攻击流量往往来自多个源头,使得单一的输入调试法难以有效应对。
其次,该方法依赖于ISP之间的合作,如果ISP之间缺乏合作意愿或能力,可能会影响回溯的准确性和效率。
最后,输入调试法的跟踪系统只在攻击时运行,这意味着在攻击未发生时,系统可能无法提供实时的网络监控和预警功能。

% 3)~控制泛洪法\par
控制泛洪法的工作原理在于受害者利用预定义的ISP地图,通过迭代方式将数据包淹没至其上游路由器,并在此过程中观察攻击强度的变化。
这一递归过程使得受害者能够在每个上游级别上逐步揭示攻击源的位置。
该方法的优点主要体现在两个方面。
首先,控制泛洪法与现有的网络协议和基础设施高度一致,无需对网络架构进行大规模改动,从而降低了实施难度和成本。
其次,该方法支持简单和增量的实现,可以根据实际需求逐步扩展应用范围,提高了灵活性。

然而,控制泛洪法也存在一些不容忽视的缺点。
首先,追踪过程仅在持续的攻击中进行,这意味着在攻击暂停或结束后,系统可能无法继续提供有效的回溯信息。
其次,该方法需要具备网络拓扑结构的先验知识,对于不熟悉网络结构的用户或组织来说,可能会存在一定的实施难度。
最后,控制泛洪法对分布式拒绝服务(DDoS)攻击无效,因为DDoS攻击往往来自多个源头,难以通过单一的淹没策略进行追踪。

2)~消息传递法\par
消息传递在传输回溯相关信息方面展现出了其独特的灵活性,特别是在采用Taylor等人提出的基于互联网控制消息协议(ICMP)的方案时,这种优势更加明显\cite{sikos2020packet}。
该方案的核心思想在于,每个路由器都会以一定的概率生成一种称为跟踪包或iTrace消息的ICMP包。
这些包携带了丰富的信息,如下一跳和上一跳信息、时间戳、MAC地址等,这些信息对于回溯过程至关重要。在攻击发生期间,大量的iTrace数据包能够有效地支持回溯操作,帮助定位攻击源。
然而,消息传递方法并非没有挑战。为了避免这些ICMP消息造成的网络流量拥塞,通常需要将消息生成的概率控制在可接受的范围内。这需要在确保回溯效率和减少网络负载之间找到一个平衡点。
Hsu和Chiueh\cite{Hsu2003TrafficSourceIdentification}以及Wang和Schulzrinne\cite{WangSchulzrinne2004DoS,WangSchulzrinne2004ReflectiveDoS}的研究工作,就为研究者提供了此类方法的实践案例和参考。

图\ref{fig:messaging}展示了消息传递法的基本原理:通过在网络中传播携带关键信息的ICMP包,实现对攻击路径的追踪。
\begin{figure}[h]
  \centering
  \includegraphics[width = 0.8\textwidth]{messaging.drawio.png}
  \caption{消息传递法原理}
  \label{fig:messaging}
\end{figure}
这种方法不仅支持具有低ISP合作能力的增量部署,而且与现有的协议和基础设施保持高度一致,从而确保了实施的可行性。
此外,由于ICMP消息的持久性,它还允许对攻击后进行详细的分析,为后续的防御措施提供有力支持。

然而,消息传递方法也存在一些明显的缺点。
由于缺乏有效的身份验证机制,这些ICMP消息可能会被攻击者利用,进行伪造或篡改,从而影响回溯的准确性。
此外,生成额外的ICMP数据包必然会增加网络流量开销,特别是在大规模网络或高流量场景下,这种开销可能会变得尤为显著,不容忽视。\par

3)~路由器日志记录法\par
记录法是一种在中间路由器上存储数据包摘要的技术,它允许受害者服务器在遭受攻击时,通过重构攻击路径来确定网络路径。
日志记录法的原理如图~\ref{fig:loggging}~所示。
\begin{figure}[htbp]
  \centering
  \includegraphics[width = 0.8\textwidth]{logging.drawio.png}
  \caption{路由器日志记录法原理}
  \label{fig:loggging}
\end{figure}
这种方法的一个显著优点是它与协议和现有基础设施高度兼容,使得在实际网络环境中能够顺利应用。
此外,记录法还支持攻击后分析,使得受害者在攻击发生后仍有机会追溯攻击来源。
更重要的是,由于它可以追溯单个数据包,因此能够提供非常详细的攻击路径信息。\par

然而,这种方法也面临着一些挑战。
由于需要在路由器上存储大量的数据包摘要,记录法会带来巨大的存储开销,对路由器的内存和处理能力提出了很高的要求。
此外,由于需要ISP的合作来收集和存储这些信息,隐私问题也成为了这种方法的一个难点。
最后,记录法需要及时进行追踪,因为路由器会定期刷新以前记录的信息,这就要求受害者必须在攻击发生后尽快采取行动。
为了克服这些挑战,研究者们提出了一些改进方法。
例如,Snoeren等人\cite{Snoeren2001}提出了一种基于哈希的IP追踪方法,称为源路径隔离引擎,该方法利用布隆过滤器等空间高效的数据结构,显著减少了路由器存储数据包摘要的存储开销。
随后,Kai等人\cite{Kai2009}和Hilgenstieler等人\cite{Hilgenstieler2010}进一步对此方法进行了改进,增强了其性能。
这些改进使得记录法在实际应用中更加可行和有效。\par

4)~覆盖法\par
覆盖网络主要由用于追踪路由的专用路由器构成,这些路由器被称为跟踪路由器,它们肩负着监控流量的重任。
覆盖网络的原理如图~\ref{fig:overlay}~所示。
\begin{figure}[htbp]
  \centering
  \includegraphics[width = 0.8\textwidth]{overlay.drawio.png}
  \caption{覆盖法原理}
  \label{fig:overlay}
\end{figure}
一旦检测到攻击,覆盖网络会迅速响应,引导流量通过这些专用的跟踪路由器。
这些路由器会仔细检查流经的流量,并提取用于追踪攻击来源的关键信息。
Stone提出的CenterTrack系统,就是这样一个典型的例子。
该系统通过分析流经中心化跟踪路由器的流量,为用户提供高效的追踪服务\cite{stone2000centertrack}。
此外,Castelucio等人\cite{castelucio2009aslevel}和Tian等人\cite{tian2011easytrace}的工作也采用了类似的基于覆盖网络的方法进行IP追踪。\par

这种方法的优势在于能够提供准确的追踪结果,有效应对DDoS攻击,并且客户端无需参与追踪过程,大大减轻了用户的负担。
然而,它也存在一些明显的缺点。
首先,实施这样的系统需要较高的成本,包括硬件设备的采购、网络的部署以及维护等费用。
其次,这种方法缺乏对渐进式部署的支持,可能需要在短时间内进行大规模的改动,对现有的网络架构造成较大的冲击。
最后,跟踪路由器作为关键节点,其本身也可能成为DDoS攻击的目标,一旦遭受攻击,整个追踪系统可能陷入瘫痪。


5)~模式分析法\par
这类方法以Xiaofeng等人\cite{xiaofeng2004mechanism}、Chen等人\cite{chen2006tracing}和Lai等人\cite{lai2008antbased}所做出的工作为代表。
这种方法的核心在于所有参与溯源的路由器通过分布式的方式协同工作,共同收集流量信息并追踪攻击源。
该方法具有许多优点,例如客户端无需参与追踪过程,这大大减少了受害者的操作负担;同时,通过分布式处理追踪过程,系统能够更高效地利用网络资源,提供了更好的可扩展性。
然而,其缺点也不容忽视。
持续的流量监控导致路由器处理开销显著增加,可能影响到网络的正常运行。
特别是在DDoS攻击场景下,随着攻击流量的急剧增加,这种方法的复杂性也随之上升,可能导致溯源过程变得更为困难。


6)~数据包标记法\par
数据包标记法背后的关键思想是利用数据包本身来记录路由信息,这一技术为追踪网络攻击提供了全新的视角。
其原理如图\ref{fig:packet_marking}所示,
\begin{figure}[h]
  \centering
  \includegraphics[width = 0.8\textwidth]{packetmarking.drawio.png}
  \caption{数据包标记法原理}
  \label{fig:packet_marking}
\end{figure}
标记数据包可以携带完整的编码路由信息或由中间路由器嵌入的一个或多个标记,这些信息为受害者提供了探索数据包所遍历路径的重要线索。
通过识别收集到的标记数据包,并利用存储在标记字段中的信息,受害者能够跟踪攻击源,从而有效地应对网络威胁。
此外,数据包标记法有三种主要的标记策略:节点附加法、节点采样法、边采样法\cite{fazio2020packet}。

节点附加法,也称确定性标记法。
在节点附加法中路由器的信息将会作为标记被逐个追加到原始IP包头中\cite{suresh2021sanguine,Nur2021Abdullah}。
图~\ref{fig:ipv4_header}~展示了数据包头部字段被用来填充标记信息的使用情况。
\begin{figure}[h]
  \centering
  \includegraphics[width = \textwidth]{ipv4_header.png}
  \caption{标记过程中使用的IPv4头部字段}
  \label{fig:ipv4_header}
\end{figure}
这种机制通过在单个数据包中携带多个路由信息简化了标记过程。
因此这种方法仅仅依靠单个标记包便能确定一条完整的路径。
但此种方法由于增加了过多的标记字段导致高网络带宽开销。
这一缺陷也限制了此种方法的应用。\par

节点采样法,也称概率性标记法。
在此方法中,每个标记数据包仅能携带一个路由节点的信息。
因此,每当数据包经过一个路由器,该路由器便会将自己的路由信息写入数据包,从而覆盖先前的信息。
这种做法不仅限制了数据包能携带的信息量,还降低了标记策略的灵活性。
此外,在节点采样法中,虽然通常使用路由器的IP地址作为标记信息,但也有多种创新方案可供选择。
例如,可以为路由器分配颜色、身份编号或增加其他特定功能来进行标记,这些多样化的标记方法不仅增加了信息量,还极大地提升了标记策略的灵活性。
这些方案在相关文献如\cite{Zhou2019Linna,Wu2018Bo}中也有深入探讨。\par

边采样法涉及到边缘信息的编码,如起始、结束等,而不是再针对单个节点信息的操作。
其他常见的边缘信息属性包括边缘权重、颜色、身份编号等。除了这些属性外,
巧妙地使用距离字段,还能够让方案在不需要预先了解互联网拓扑的情况下具备构建攻击路径的能力。\par

尽管数据包标记法存在一些缺点,如数据包头部字段可能过载、可能产生误报或需要修改网络协议等,但其优点仍使其在网络攻击追踪中占据重要地位。
数据包标记法的核心优势在于与现有网络基础设施的高度兼容性,这避免了大规模改动网络架构的需要,从而显著降低了实施难度和成本。
此外,该方法实施起来简单灵活,能够快速定位攻击源,为受害者提供及时的保护。
特别值得一提的是,数据包标记法在应对DDoS攻击方面表现出色,这使得它在网络安全领域具有广阔的应用前景。\par

这几种方法的性能表较如图~\ref{tb:compare_approch}~所示。
\begin{table}[h]
  \caption{IP回溯方法性能比较}
  \label{tb:compare_approch}
  \centering
  \setlength{\tabcolsep}{1.5pt}
  \begin{tabular}{ccccccc}
  \toprule
  {\heiti 回溯方法} & {\heiti 管理开销} & {\heiti 网络开销} & {\heiti 路由器开销} & {\heiti ISP合作} & {\heiti 假阳率} & {\heiti 时间开销}\\
  \midrule
  链路测试法 & Moderate &  Low & Moderate & Collaborative & High & Long\\
  数据包标记法 & Low & Low & Low & Noncollaborative & Low & Long\\
  路由器日志记录法 & High & Low & High & Collaborative & Low & Middle\\
  消息传递法 & Low & Low & Low & Noncollaborative & Very Low & Long\\
  覆盖法 & Low & Low & High & Collaborative & Low & Middle\\
  模式分析法 & Low & Moderate & High & Collaborative & Low & Middle\\
  \bottomrule
  \end{tabular}
  \end{table}

\section{论文研究内容}

尽管现有的网络攻击检测技术已经取得了长足的进步,但在实际应用中仍然暴露出不少短板。
首要问题在于,现有的检测模型大多仅聚焦于流量数据的单一维度特征,这在面对日益复杂且多维度的流量数据时显得力不从心,导致模型的检测效能大打折扣。
为了弥补这一不足,本文充分利用残差网络以及双向门控循环单元在提取与处理时空特征方面的优势,并结合多模态特征融合技术,提出了一种基于残差网络与双向门控循环单元的多模态特征融合模型。
此外,当前的网络攻击检测技术多依赖于静态数据集,这在快速变化的网络环境中显得捉襟见肘,实际应用中面临诸多挑战。
为了应对这一挑战并赋予模型增量学习的能力,本文为模型引入了iCaRL技术并基于遗传算法对该技术进行了优化改进。
最后,为了确保模型在面对DDoS攻击时能够稳定运行,本文提出了一种基于路由器接口号成组标记的溯源方法。
基于这一方法,本文进一步实现了基于溯源的模型防护技术。
因此,本文的研究内容及主要贡献如下:\par

1)~鉴于当前网络攻击检测技术主要关注流量数据的单一维度,特征挖掘不足,因此异常流量检测精准度仍有提升空间。
为此,本文创新性地提出了一种基于残差网络与双向门控循环单元的多模态特征融合模型。
该模型结合残差网络的强大空间特征提取能力与双向门控循环单元的时序特征分析优势,通过多模态特征融合技术,全面深入地分析流量数据的时空特征,进而提升检测精准度。\par

2)~针对当前的网络攻击检测技术大多依赖于静态的数据集,难以适应快速变化的网络环境的问题,
本文引入了iCaRL技术并基于遗传算法对该技术进行了优化改进,使得所提模型能够在保留原有知识的基础上,迅速学习并适应新的攻击模式,确保网络安全的持续性和稳定性。\par

3)~为应对DDoS攻击对模型可能造成的严重负担,本文基于数据包标记溯源原理,提出了一种基于路由器接口号成组标记的溯源方法。
该方法充分利用路由器接口号的局部性特点,在不增加数据包大小的情况下,对路径上的一组连续路由器进行有效标记。             
这使得单个数据包能够携带更为丰富的路由信息,从而迅速且准确地完成路径溯源工作。
最后,本文在该溯源方法的基础上,设计并实现了基于溯源的防护方法。
一旦定位到攻击者源头,该方法便能从源头处直接过滤掉攻击流量,有效减轻模型的负担,确保网络的安全稳定。\par

\section{论文组织结构}
本论文一共有5个章节,其组织结构如图~\ref{fig:paper_structure}~所示。
\begin{figure}[h]
  \centering
  \includegraphics[width = 0.85\textwidth]{paper_structure.drawio.png}
  \caption{论文组织结构}
  \label{fig:paper_structure}
\end{figure}

第~\ref{cha:overview}~章,绪论。
在本章主要介绍了本文的研究,即基于特征融合的异常流量检测模型及防护方法研究,所处的背景以及具备的研究意义。
之后,本文详细地介绍了网络溯源技术的研究现状。
最后交代了本文的研究内容及主要贡献点以及论文的组织架构。


第~\ref{cha:basic-knowledge}~章概述了残差神经网络、双向门控循环单元网络、遗传算法、伪造源地址攻击与数据包标记法的相关理论基础。

第~\ref{cha:ResNet-BiGRU}~章首先提出了基于残差网络与双向门控循环单元的多模态特征融合模型,并通过实验验证了其优越性。
接着本文利用此模型并结合iCaRL技术,实现了增量学习方案。
为进一步优化该方案的性能,本文又提出了一个基于遗传算法的记忆集抽样优化策略。
最后利用UNSW-NB15和NSL-KDD数据集分别作为新旧任务数据集对该方案进行了实验验证,实验结果表明该方案能有效支持模型增量学习和知识保留。

第~\ref{cha:IGPPM}~章,基于数据包标记溯源的模型防护方法。
本章首先提出了基于路由器接口号成组标记的溯源方法,并介绍了该方法的原理及流程。
随后,本章通过几组对比实验,展示了该溯源方法在溯源准确率和溯源速度方面的性能表现。
接着,本章在该溯源方法的基础上,设计并实现了基于溯源的模型防护方法。
最后为了验证该防护方法的实用性,本章创建了一个高度仿真的网络攻击环境,并对该防护方法进行了实际的测试。

第~\ref{cha:over_view}章,总结和展望。
总结了本文的所有研究工作以及取得的成果,最后指出了本文未来的研究内容和方向。

最后是参考文献、攻读硕士学位期间发表的学术论文目录和致谢。
