%%================================================
%% Filename: chap02.tex
%% Encoding: UTF-8
%% Author: Yuan Xiaoshuai - yxshuai@gmail.com
%% Created: 2012-04-27 19:37
%% Last modified: 2019-11-05 23:57
%%================================================
\chapter{相关理论知识与关键技术}
\label{cha:basic-knowledge}

\section{残差神经网络}
残差神经网络(Residual Neural Network,简称ResNet)是一种深度神经网络架构,它在2015年被提出,并且因其出色的性能获得了ImageNet大规模视觉识别挑战(ILSVRC)\footnote{http://image-net.org/challenges/LSVRC/2015/ 和 http://mscoco.org/dataset/\#detections-challenge2015.}的胜利\cite{he2016deep}。
ResNet的提出,解决了深度网络训练过程中的梯度消失以及梯度爆炸问题,使得训练更深层次的网络模型成为可能,从而极大地提高了图像识别、分类任务的准确率。

在传统的深度神经网络中,随着网络层数的加深,训练难度逐渐增大。
这主要是因为梯度在反向传播过程中会逐渐消失或爆炸,导致网络权重难以得到有效更新。
为了解决这个问题,ResNet引入了“残差学习”的概念,即通过添加跳跃连接(Shortcut Connections)将输入信息直接传递到更深层的网络中,为信息流动提供了一条更短的路径。

ResNet由多个残差块组成,每个残差块内部包含多层卷积层和一个跳跃连接(图~\ref{fig:resnet_structure}展示了一个由两层卷积层堆叠而成的残差块结构)。
\begin{figure}[h]
  \centering
  \includegraphics[width = 0.6\textwidth]{resnet_block.drawio.png}
  \caption{两层堆叠的残差块结构图}
  \label{fig:resnet_structure}
\end{figure}
通过跳跃连接,ResNet能有效地解决梯度消失和爆炸问题,使得训练深度模型成为可能。


跳跃连接的原理在于,ResNet不再要求网络层直接拟合一个期望的底层映射,而是让网络层拟合一个残差映射。
假设期望的底层映射为H(x),那么网络层需要拟合的残差映射则为:
\begin{equation}
  \label{eq:residual_mapping}
  F(x) := H(x) - x
\end{equation}
因此,原始映射可以重新表述为F(x)+x。
这种优化方式使得网络更容易学习到恒等映射,从而确保增加的层不会降低网络的性能。
通过这种方法,研究人员便可以构建更深的网络模型而不会降低性能,这在许多图像视觉识别任务中都达到了前所未有的准确率。


此外,ResNet的应用范围不仅局限于图像分类、检测和识别任务。
其强大的特征提取能力和优秀的性能表现使得它在自然语言处理、音频分析等其他领域也得到了广泛的应用。
可以说,ResNet的提出为深度学习领域的发展注入了新的活力。

\section{双向门控循环单元}
传统的循环神经网络(RNN)在处理长序列数据时,往往会遭遇梯度消失和梯度爆炸的挑战。
梯度消失导致网络难以捕获长期依赖关系,因为随着序列长度的延伸,早期时间步的信息对后续时间步的影响逐渐减弱,仿佛被“遗忘”。
而梯度爆炸则可能引发训练过程的不稳定,使得模型难以收敛。
为了克服这些限制,研究者们提出了更为先进的循环网络结构,如长短期记忆网络(LSTM)\cite{2023Comparative}和门控循环单元(GRU)\cite{2023Short}。
这些结构通过精心的设计,显著增强了网络对长期依赖关系的捕捉能力。
LSTM 通过门控机制使循环神经网络不仅能记忆过去的信息,同时还能选择性地忘记一些不重要的信息而对长期语境等关系进行建模。
同样地,GRU也是基于这种思路设计的,旨在更好地保留长期序列信息并有效减轻梯度消失的问题。
相比LSTM,使用GRU能够达到相当的效果,并且相比之下更容易进行训练,能够很大程度上提高训练效率,因此很多时候会更倾向于使用GRU。

图~\ref{fig:GRUunit}~是GRU基本单元的组成结构图。
\begin{figure}[h] 
  \centering
  \includegraphics[width = 0.8\textwidth]{gru.drawio.png}
  \caption{GRU基本单元结构图}
  \label{fig:GRUunit}
\end{figure}
\begin{flushleft}
  \renewcommand\arraystretch{1.25}
  \begin{tabularx}{\textwidth}{@{}>{\normalsize\rm}l@{\quad}>{\normalsize\rm}l@{——}>{\normalsize\rm}X@{}}
图~\ref{fig:GRUunit}~中
  &  $\times$ &Hadamard Product,矩阵相乘运算;\\
  &  $+$ &矩阵加法运算;\\
  &  $σ$ &Sigmoid 激活函数,输出范围为0 -- 1;\\
  &  $\tanh$ & $\tanh$激活函数,输出范围为-1 -- 1;\\
  \end{tabularx}\vspace{.5ex}%TODO : 注释内容自动转页接排
\end{flushleft}
GRU由重置门(reset gate)和更新门(update gate)两种类型的门控机制构成。
这些门控制着信息的流动,决定了哪些信息应当被保留,哪些信息应当被忽略或删除,从而有效地捕捉到长期依赖性。\par

其中,重置门决定了多少过去的信息需要被忘记。
它允许模型抛弃与新信息无关的旧状态信息,使模型能够更灵活地处理每个时间点的数据。
GRU得到门控信号$r_t$之后,首先使用重置门控来得到“重置”之后的数据$h'_{t-1} = r_t \cdot h_{t-1}$,再将$h'_{t-1}$与输入$x_t$进行拼接,再通过一个$\tanh$激活函数来将数据放缩到-1 -- 1的范围内。
这一过程如公式~\ref{eq:gru_rt}、\ref{eq:gru_ht}~所述。
\begin{gather}
  r_t = σ(W_r \odot [h_{t-1},x_t]) \label{eq:gru_rt} \\
  \tilde{h_t} = \tanh(W \odot [r_t \cdot h_{t-1} , x_t]) \label{eq:gru_ht}
\end{gather}
这里的$h'_{t-1}$主要是包含了当前输入的$x_t$数据。
有针对性地对$h'_{t-1}$添加到当前的隐藏状态,相当于“记忆了当前时刻的状态”。
类似于LSTM的选择记忆阶段。
更新门帮助模型决定过去的信息有多少需要保留到未来。
它在决定状态信息是否更新时起着关键作用,类似于LSTM中的遗忘门和输入门的组合。
“更新记忆”阶段是GRU最关键的一个阶段。
在这个阶段,GRU同时进行了遗忘与记忆两个步骤。
使用先前得到的更新门控$z$便得到更新表达式: 
\begin{equation}
  \label{dq:gru_ht}
  h_t = (1-z) \odot h_{t-1} + z \odot h'_{t-1}
\end{equation}
门控信号$z$的范围为0 -- 1。门控信号越接近1,代表“记忆”下来的数据越多;而越接近0则代表“遗忘”的越多。
可以看到,这里的遗忘$z$和选择$(1-z)$是联动的。
也就是说,对于传递进来的维度信息,模型会进行选择性遗忘,遗忘了多少权重($z$),GRU就会使用包含当前输入的$h'_{t-1}$中所对应的权重进行弥补,以保持一种“恒定”状态。

% \subsubsection*{双向门控循环单元}

尽管GRU能够有效地处理序列数据中的长期依赖关系,并通过门控机制来控制信息的流动,但它在某些任务中仍然存在一些挑战。
具体而言,GRU只能按照序列的时间顺序依次处理每个时间步的信息,这意味着在处理某个时间步时,它只能利用该时间步之前的信息,而无法同时利用该时间步之后的信息。
然而,在某些任务中,模型需要同时考虑序列中过去和未来的信息。
例如,在语音识别或自然语言处理中,当前词的识别或理解可能不仅取决于它之前的词,还取决于它之后的词。
在这种情况下,单向的GRU可能无法充分捕获序列中的上下文信息。
为了解决这个问题,可以引入双向门控循环单元(Bi-directional GRU,BiGRU)。
图~\ref{fig:BiGRU}~展示的是一个沿着时间展开的双向循环神经网络。
\begin{figure}[h]
  \centering
  \includegraphics[width = 0.8\textwidth]{BiGRU.drawio.png}
  \caption{BiGRU结构图}
  \label{fig:BiGRU}
\end{figure}
与单向的GRU不同,双向GRU在每个时间步都会同时考虑正向和反向的序列信息。
具体来说,它会分别从前向和后向两个方向对序列进行处理,然后将两个方向上的隐状态进行拼接或求和等操作,以得到每个时间步的最终输出。\par

BiGRU包含两个GRU层,一个沿正时间方向处理数据(正向GRU),捕捉从过去到现在的依赖;
另一个沿反时间方向处理数据(反向GRU),捕捉从未来到现在的依赖。
这两个层的输出通常会在每个时间步被合并,以提供给定时间点前后的完整上下文信息。
另外,从GRU继承的更新门和重置门可以使模型在每个方向上更有效地学习长期和短期依赖性。

对于一个输入序列 $X = x_1 x_2 \dots x_T$,BiGRU模型的前向计算可以表示为:
\begin{gather}
  \overleftarrow{h_t} = GRU_1(x_t,\overleftarrow{h_{t-1}}) \label{eq:bigru_htleft} \\
  \overrightarrow{h_t} = GRU_1(x_t,\overrightarrow{h_{t-1}}) \label{eq:bigru_htright} \\
  y_t = \overleftarrow{h_t} \oplus \overrightarrow{h_t} \label{eq:bigru_out}
\end{gather}
\begin{flushleft}
  \renewcommand\arraystretch{1.25}
  \begin{tabularx}{\textwidth}{@{}>{\normalsize\rm}l@{\quad}>{\normalsize\rm}l@{——}>{\normalsize\rm}X@{}}
  其中
  &  $y_t$ &模型输出;\\
  &  $\overrightarrow{h_t}$ &前向输出;\\
  &  $\overleftarrow{h_t} $ &后向输出;\\
  &  $\oplus$ & element-wise sum按位加。\\
  \end{tabularx}\vspace{.5ex}%TODO : 注释内容自动转页接排
\end{flushleft}

\section{遗传算法}
\label{sec:GA}
\subsection{遗传算法概述}
遗传算法(Genetic Algorithm,简称GA)\cite{alhijawi2023genetic,gen2023genetic}是一种受生物进化理论启发的搜索启发式算法。
它通过模拟自然界中的遗传(Heredity)、变异(Mutation)、交配(Crossover )和选择(Selection)机制,有效解决优化和搜索问题。\par

算法起始于一个随机生成的初始种群,种群中的每个个体代表问题空间的一个潜在解,并以特定方式(如数值或结构形式)编码。
随后,算法评估每个个体的适应度,以确定其在给定问题中的效用。
接下来,遗传算法利用遗传操作符选择最适应的个体进行交叉和变异操作,从而产生新一代种群。
这一过程循环进行,直至满足终止条件,如达到预设的迭代次数或解的质量达到要求。

由于遗传算法对初始解的质量不敏感,且能在复杂、多峰值的搜索空间中找到全局最优解,因此它在优化问题、机器学习、调度和人工智能等领域得到了广泛应用。
作为基于进化论和遗传学机理的搜索算法,遗传算法涉及一些生物遗传学知识,下面是遗传算法中的一些常用术语。
\begin{enumerate}[label=\arabic*)] 
  \item 种群(Population):一个种群由多个个体组成,每个个体代表了问题空间中的一个可能解。

  \item 个体(Individual):在遗传算法中,个体通常用一个字符串(最常见的是二进制串)表示,代表了问题的一个潜在解决方案。
  
  \item 基因(Gene):个体表示中的一个元素(如二进制串中的一位),代表解决方案的一个特征。
  
  \item 染色体(Chromosome):个体的完整表示,即一组基因的组合,代表了一个完整的解决方案。
  
  \item 适应度(Fitness):一个函数,用于评估个体的适应环境的能力,即解决方案的好坏。适应度越高,个体被选中的机会越大。
  
  \item 选择(Selection):在当前的种群中,选择个体进行繁殖的过程通常是根据个体的适应度来进行的。在这个过程中,具有较高适应度的个体会被赋予更高的选择机会,以确保优良基因的传递和种群的进化。
  
  \item 交叉(Crossover):也称为杂交,是一个遗传操作,其中两个个体交换它们的一部分基因,以产生新的后代。这模仿了生物遗传中的性繁殖过程。
  
  \item 变异(Mutation):在遗传算法中,变异是指随机改变个体的染色体中的一些基因,以引入新的遗传多样性。这可以帮助算法避免局部最优解,探索更广泛的搜索空间。
  
  \item 代(Generation):遗传算法的一个迭代步骤,在其中通过选择、交叉和变异操作创建一个新的种群。
  \end{enumerate}

  基本遗传算法(也称标准遗传算法或简单遗传算法,Simple Genetic Algorithm,简称SGA)是一种群体型操作,该操作以群体中的所有个体为对象,只使用基本遗传算子(Genetic Operator):选择算子(Selection Operator)、交叉算子(Crossover Operator)和变异算子(Mutation Operator)。
  SGA的表示方法为:
  \begin{equation}
    \label{eq:GA}
    SGA = (C, E, P_0, M, \phi, \Gamma, \psi,T)
  \end{equation}
  \begin{flushleft}
    \renewcommand\arraystretch{1.25}
    \begin{tabularx}{\textwidth}{@{}>{\normalsize\rm}l@{\quad}>{\normalsize\rm}l@{——}>{\normalsize\rm}X@{}}
    式中
    
    &  $C$ &个体的编码方案;\\
    &  $E$ &个体适应度评价函数;\\
    &  $P_0$   &初始种群;\\
    &  $M$ & 种群大小;\\
    &  $\phi$ & 选择算子;\\
    &  $\Gamma$ & 交叉算子;\\
    &  $\psi$ & 变异算子;\\
    &  $T$ & 遗传算法终止条件。\\
    \end{tabularx}\vspace{.5ex}%TODO : 注释内容自动转页接排
  \end{flushleft}

    \subsection{遗传算法步骤}
    1)~染色体编码\par
    在遗传算法中,将问题的可行解从其原始解空间转换到算法可以搜索的基因型空间的过程称为编码。
    简而言之,编码就是将问题的解转换成遗传算法中的染色体结构,这些染色体由一系列的基因型串组成,不同的基因型串组合代表解空间中不同的点。
    在遗传算法开始搜索最优解之前,必须先完成这种从解到基因型的映射。遗传算法中常用的编码方式有二进制编码、实数编码、置换编码、值编码。\par
    
    
    二进制编码:这是最常见的编码方式,其中每个染色体由一串二进制数字(0和1)组成。每个数字可以看作是一个基因,整个字符串表示一个个体。二进制编码容易实现交叉和变异操作,但可能不适用于所有类型的问题。

    实数编码:对于需要连续值的优化问题,染色体可以由实数(浮点数)序列组成。实数编码更适合处理那些参数自然为实数的问题。

    值编码:在某些问题中,可以直接使用问题域中的值来编码染色体。
    例如,如果问题涉及配置一组参数,每个参数可以取自一个预定义的集合,那么染色体可以是这些参数值的一个序列。

    置换编码:这种编码方式适用于排列问题,如旅行商问题(TSP)\footnote{旅行商问题(Traveling Salesman Problem, TSP)是一个经典的优化问题,它寻求最短的路径让旅行商访问一系列城市并返回出发点,每个城市只能访问一次。}。
    在这种编码中,染色体是一组数字的排列,表示解决方案中元素的顺序。

    例如,如果用长度为$k$位的二进制编码来表示一个参数,其取值范围在 $[a, b]$ 之间,那么可以得到 $2^k−1$ 种不同的编码,参数编码的对应关系为:
    \begin{equation}
      \label{eq:encode}
      \begin{aligned}
        000000\dots0000 &= 0 \quad &\rightarrow a \\
        000000\dots0001 &= 1 \quad &\rightarrow a + \delta \\
        000000\dots0010 &= 2 \quad &\rightarrow a + 2\delta \\
        &\vdots \\
        111111\dots1111 &= 2^k - 1 \quad &\rightarrow a + (2^k - 1)\delta \rightarrow b
      \end{aligned}
    \end{equation}    
    其中,$\delta = \frac{b-a}{2^k - 1}$

    2)~染色体解码\par
    编码的逆操作,将基因型空间中的染色体转换回原始问题的解空间的过程,即从算法表示的解(通常为基因串或染色体)提取出问题的具体可行解。
    例如,已知编码规则如上式~\ref{eq:encode}~,则其对应的解码公式为:
    \begin{equation}
      \label{eq:decode}
      X = a + (\sum\limits_{i=1}^{k}b_i \cdot 2^{i-1}) \cdot \frac{b-a}{2^k-1}
    \end{equation}

    3)~初始群体的生成\par

设定最大进化代数 \(T\),群体规模 \(M\),交叉概率 \(P_c\),变异概率 \(P_m\)。随机生成 \(M\) 个个体以形成初始群体\(P_0\)。

    4)~适应度值评估检测\par
适应度函数用于评估解或个体的表现好坏,不同问题需通过不同的适应度函数来定义。基于特定问题,需对群体$P(t)$内每个成员的适应度进行计算。
适应度函数的调整通常意味着在算法进化过程的不同阶段,通过调整适应度值的大小来避免因群体内适应度过于接近而减弱竞争,避免算法收敛到局部最优。
常见的适应度调整策略包括线性调整、幂次调整和指数调整。

  线性调整:
  线性调整通过公式~\ref{eq:ga_trans1}~将原始适应度$F$进行线性转换,以调节个体间的适应度差异,其中$a$是比例系数,$b$是平移系数。
    \begin{equation}
      \label{eq:ga_trans1}
      F' = aF + b
    \end{equation}
    
    
    
    幂次调整:
    幂次调整通过公式~\ref{eq:ga_trans2}~对原始适应度$F$进行调整,使用幂次$k$来放大或缩小适应度值,以改变个体间的适应度差异。
    \begin{equation}
      \label{eq:ga_trans2}
      F' = F^k
    \end{equation}
    
    指数调整:
    指数调整通过公式~\ref{eq:ga_trans3}~对原始适应度$F$进行调整,利用指数函数和参数Γ来显著改变适应度分布,从而调节个体间的竞争强度。
    \begin{equation}
      \label{eq:ga_trans3}
      F' = e^{-\Gamma F}
    \end{equation}

5)~遗传算子\par
选择:
选择机制通过挑选现有种群中的较优个体构建新一代种群,进而促进后续代的进化。
一个个体的选择机会与其适应度成正比,适应度越高的个体,其成为下一代的父母的几率也越高。
以轮盘赌选择法为例,设定种群总数为M,第i个个体的适应度记作$f_i$,其被选为下一代的概率可以表示为:
\begin{equation}
  \label{eq:selection}
  P_i = \frac{f_i}{\sum\limits_{k=1}^{M} f_k}
\end{equation}
确定个体被选中的几率后,通过生成一个[0,1]范围内的随机数来选出参与繁殖的个体。
如果一个个体的选中几率较高,它可能会被选中多次,导致其基因在种群中广泛传播;反之,选中几率低的个体可能会被逐渐淘汰。

交叉:
交叉过程从群体中随机挑选出两个个体,然后通过交换它们的基因序列来传递优秀的遗传特性给下一代,创造性能更好的新个体。
在遗传算法的实践中,单点交叉算子是最常用的方法,其具体执行过程如图~\ref{fig:onepoint_corss}~所示。
\begin{figure}[h]
  \centering
  \includegraphics[width = 0.25\textwidth]{onepoint_cross.drawio.png}
  \caption{单点交叉流程图}
  \label{fig:onepoint_corss}
\end{figure}
单点交叉在两个个体的染色体上随机选定一个点作为交叉点,然后交换这个点之后的基因段。
其他类型的交叉操作还有双点和均匀交叉技术。双点交叉在匹配的染色体上随机确定两个位置,并将这两个位置之间的染色体段进行交换,以此来调整基因序列。
均匀交叉则对配对染色体上的每一个基因位都以相同的概率进行交叉,生成新的基因组合。
算术交叉通过对配对染色体执行线性组合的交叉,以产生变化的基因序列。
\begin{figure}[h]
  \centering
  \includegraphics[width = 0.9\textwidth]{crossshow.drawio.png}
  \caption{交叉操作示意图}
  \label{fig:cross_show}
\end{figure}

变异:
为避免遗传算法过早收敛至局部最优,变异操作被引入以增强搜索的多样性。在应用中,常用的是单点变异,亦称位变异。
单点变异随机选取染色体上的一个基因位,并反转其值:若为二进制编码,则将0改为1,1改为0。
\begin{figure}[h]
  \centering
  \includegraphics[width = 0.9\textwidth]{mutate.drawio.png}
  \caption{变异操作示意图}
  \label{fig:mutate}
\end{figure}

群体$P(t)$经过选择、交叉、变异运算后得到下一代群体$P(t+1)$。
若$t\leq T$,则 $t \leftarrow t + 1$,重新计算适应值;否则以进化过程中所得到的具有最大适应度的个体作为最好的解输出,终止运算。
图~\ref{fig:ga_procedure}~是整个遗传算法的流程图。
\begin{figure}[h]
  \centering
  \includegraphics[width = 0.8\textwidth]{ga_procedure.drawio.png}
  \caption{遗传算法流程图}
  \label{fig:ga_procedure}
\end{figure}

\section{源地址伪造攻击的分类}

源地址伪造攻击可以根据伪造产生的虚假地址与攻击者受害者所在网络的关系来进行分类,总共可分为六类:
a.虚假地址在网络中不存在或已经失活;
b.虚假地址指向的主机是受害者;
c.虚假地址与受害者主机处于同一个子网中;
d.虚假地址与攻击者处于同一个子网中;
e.虚假地址处于攻击者和受害者之间的路径中;
f.虚假地址既不在攻击者与受害者的子网中也不在攻击者与受害者之间的路径当中\cite{lee2021classification}。
以下是这六类攻击的详细描述。

1)~a类\par
恶意主机通过产生随机的互联网中不存在的或着已经失效的IP地址来对攻击数据包的源地址进行伪造。
产生的攻击数据包占用着受害者主机的资源,使受害者不能再向其他主机提供服务。
这些IP地址包括RFC1918中指定的私有IP地址、RFC3330中的自动分配的IP地址、循环测试地址,它们均不会在互联网中出现。
SYN Flood\cite{rahouti2021synguard,dimolianis2021syn}是一种常见的攻击方式,它能够不断消耗受害者主机CPU资源以及内存资源并阻止其他合法用户的连接。
在不设置保护策略的情况下,少量的SYN洪水攻击便足以使受害者主机崩溃。
在遭受攻击时,受害者主机会尝试向恶意数据包中的伪造源地址发送SYNACK包,以期对方返回RTS包来释放半连接。
正因如此,攻击者往往会使用前述的伪造或无效IP地址来隐匿真实身份,从而加大追踪和防范的难度。\par
2)~b类\par
攻击者将受害者主机的IP地址作为攻击数据包的源地址以此来实现反射性攻击、直接攻击和诱捕攻击,这类攻击通常有三种形式:
第一种形式,攻击数据包的源地址是受害者主机IP地址,目标地址是受害者主机的所在子网的广播地址。
攻击数据包被广播之后,受害者将收到大量的ACK回复,这会使受害者淹没于这些流量当中而大大地削弱了受害者的正常服务能力。
第二种形式,攻击数据包的源地址与目标地址同时是受害者主机的IP,受害者收到攻击数据包后会将其响应发送给自己,这一行为会导致受害者受到干扰而瘫痪。
著名的DrDoS\cite{baik2020multi,nuiaa2022comprehensive}是此类攻击的一个典型例子,它会使受害者的带宽或内存等资源溢出或者过载而不能使用。
TFN\cite{brooks2021distributed}和Land  attacks\cite{CERT1997}也是此类典型的例子。
第三种形式,攻击者向受害者发送数据包,其源主机/端口与目标主机/端口相同,并设置了SYN标志能够锁死受害者或使其协议栈崩溃。\par
3)~c类\par
此类攻击将受害者子网内的IP地址伪装成攻击数据包的源地址,借助受害者与伪造地址对应主机间的信任关系实施攻击。
基于TCP连接的盲IP欺骗攻击\cite{fonseca2021identifying}是这类攻击的典型例子。
TFN2K\cite{singh2020study}也是一个例子,它是TFN的下一代版本。\par
4)~d类\par
d类攻击将攻击者所处子网内的地址作为源地址,因为入口过滤的粒度通常不高便可以很容易使攻击数据包通过入口过滤。
反弹扫描\cite{CERT1997}是这类攻击的典型案例。攻击者通过伪造同一子网内邻居的源地址,诱使受害者发送响应包,进而嗅探并截获返回给邻居的流量数据。
此类攻击可用于端口扫描,当受害者的某端口处于关闭状态时,会回复RST数据包\cite{kaur2023comparative}。更为严重的是,这种攻击能巧妙地规避uRPF\cite{dhilipan2023detection}的防御机制,从而增加了网络安全的隐患。\par
5)~e类\par
e类攻击的攻击者通常强迫网络设备的源地址出现在攻击者与受害者之间的路径上。
在此过程中,攻击者能散布虚假的DNS或路由信息,并实现对网络流量的重定向\cite{Huang2006}。此类攻击因其高度的隐蔽性和破坏力,被视为极具危险性的网络攻击手段之一。\par
6)~f类\par
此类攻击无需依赖受害者与伪造地址间的特定拓扑关系。
攻击者常将多种攻击手段组合运用,如MITM攻击\cite{schrottenloher2023simplified},便是f类攻击中两种典型手法的结合。
以A、V1和V2为例,A在与V2通信时,会伪造V1的源地址;同样,在与V1通信时,A会伪造V2的源地址,从而实施攻击。\par

表~\ref{tab:source_address_spoofing}~展示了这六种攻击之间的关系。
\begin{table}[htbp]
  \caption{源地址伪造攻击分类}
  \label{tab:source_address_spoofing}
  \centering
  \begin{tabular}{cccc}
  \toprule
  {\heiti 分类} & {\heiti 虚假地址状态} & {\heiti 与攻击者关系} & {\heiti 与受害者关系}  \\ 
  \midrule
  a类 & 不存在或已经失活 & 无 & 无 \\ 
  b类 & 存在 & 无 & 指向 \\ 
  c类 & 存在 & 无 & 同子网 \\ 
  d类 & 存在 & 同子网 & 无 \\ 
  e类 & 存在 & 攻击者到受害路径内 & 攻击者到受害者路径内 \\ 
  f类 & 不存在 & 无 & 无 \\ 
  \bottomrule
  \end{tabular}
\end{table}

\section{数据包标记法}
图~\ref{fig:simple_topology}~是一个IP回溯问题的简单网络拓扑图。
\begin{figure}[htbp]
  \centering
  \includegraphics[width = 0.8\textwidth]{simple_topology.png}
  \caption{针对路由回溯问题的简单拓扑图}
  \label{fig:simple_topology}
\end{figure}
整个结构可以看作是一棵以$V$为根节点的树,$V$代表一个遭受攻击的受害者,它可能为服务器、防火墙或者入侵检测系统。
每个$A_i$是树中的叶子节点,它代表潜在的攻击源,它可能为例如攻击者主机也可能为普通的用户。
每个$R_i$则代表从这些$A_i$到$V$的中间路由器。
在这个网络中,每个$A_i$都有可能向$V$发送数据包,而这些数据包中则有潜在攻击数据包的可能性。
其中这些攻击数据包走过的路径都为攻击路径,它是一个从攻击节点$A_i$到$V$的有序唯一路由序列。
例如,假设$A_1$是一个攻击节点,它正在向$V$发送攻击数据包,这些攻击数据包走过的路径是唯一的,由$R_7$、$R_4$、$R_2$、$R_1$依次组成,则这条从$A_1$到$V$的唯一的路径就是一条攻击路径。
一个IP回溯问题就是通过一些方法和策略确定攻击路径,揭示攻击主机的真实身份和具体位置。\par

当回溯成功后便可以从攻击源的下游最近邻路由器过滤掉这些数据包以保护受害者服务器免受攻击。
在这个问题中,似乎只要从攻击数据包中提取出源IP就可以确定攻击路径,然而解决这个问题通常是很复杂的。
一个有预谋、有准备的攻击通常会做出一些手段,例如伪造攻击数据包的源IP地址、发送一些虚假信息来伪造攻击路径从而干扰溯源,使溯源任务极难完成。
为了有效应对这一问题,数据包标记法应运而生。\par

数据包标记法通过路由器在转发数据包时注入额外的标记信息,记录路由路径,从而为后续的路径重组和攻击溯源提供重要线索。
此方法通常由两部分组成,即标记过程和路径重组过程。
当部署了本方案的路由器$R_i$接收到数据包后,便会执行标记过程。
在标记过程中,路由器$R_i$通过向待转发的数据包注入额外的标记信息来记录自身路由信息。
这些标记信息类型可能为路由器IP地址也可能是其他一些控制信息类型,具体的标记信息类型根据所选的方案而异。
通常,这些信息被注入到数据包的头部字段中,图~\ref{fig:ipv4_header}~显示了数据包头部字段被用作标记字段的情况。
\begin{figure}[htbp]
  \centering
  \includegraphics[width = 0.8\textwidth]{ipv4_header.png}
  \caption{标记过程中使用的IPv4头部字段}
  \label{fig:ipv4_header}
\end{figure}
当受害者$V$成功收集到足够数量的标记数据包以满足回溯需求时,系统将进入收敛状态。\par

在收敛状态下,受害者所收集的标记数据包数量已完全能够支撑受害者主机完成路由回溯。
在传统的概率包标记法中,路由器$R_i$距离受害者$V$越远,受害者$V$收集到$R_i$所标记的数据包概率将越小。
因此,受害者$V$收集的数据包为距自身最远路由器$R_i$所标记的概率是最小的,为$p(1-p)^(d-1)$。
其中,$p$代表每个路由器在接收到数据包时进行标记的概率,而$d$则表示接收到数据包的路由器与数据包最终接收方之间的路由器跳数。
通常情况下,要达到收敛状态所需的最少数据包数量将由此概率决定,为$\frac{1}{p(1-p)^(d-1)}$。
受害者$V$一旦达到收敛状态后,便可根据这些标记信息使用一定的算法来执行路径重构过程,即受害者$V$会根据受到的标记信息执行相应算法找到攻击源。

\section{本章小结}
在本章中,本文首先深入探讨了残差神经网络与双向门控循环单元的核心架构和工作机制,为接下来将要介绍的模型提供了坚实的理论基础。
接着,本文详细阐述了遗传算法的基本原理,这为后续基于遗传算法的抽样优化方法奠定了重要基础。
随后,本文分类并分析了源地址伪造攻击的危害性,这为第四章中溯源技术的引入提供了强有力的依据。
最后,本文介绍了数据包标记法,该方法为第四章将要提出的溯源方案提供了切实可行的技术支撑。
