\chapter{总结与展望}
\label{cha:over_view}

\section{本文总结}


随着移动互联网的蓬勃发展和人工智能技术的不断革新,互联网已深入渗透到人们生活的每个角落,每日产生的网络流量数据呈爆炸式增长。
这种数据量的激增在极大丰富人们生活的同时,也带来了日益严峻的安全挑战。
针对这些挑战,本文从网络攻击溯源和异常流量检测两个层面进行了深入研究,并做出了如下工作\par
(1) 针对传统概率包标记法在网络溯源时存在的溯源准确率不高、溯源速度慢等问题,本文进行了创新改进,提出了一种基于路由器接口分组概率包标记的方案。
该方案通过三个阶段——路由器标记、数据包收集、路径重构,首先是路由器标记阶段,这一阶段的核心是利用数据包在传输过程中经过的路由器,对其路径信息进行精确标记。
通过这种方式,我们能够有效地收集到数据包在网络中的路径信息。
紧接着是数据包收集阶段,这一阶段的主要任务是提取已标记的数据包中的关键信息,并触发相应的路径重构程序。
通过这一阶段,我们能够获取到足够的数据来支持后续的路径分析。
最后是路径重构阶段,这一阶段充分利用异或运算的特性,从收集到的数据包信息中解析出具体的路由器接口号序列路径。
这一序列是定位数据包发送方的关键线索。一旦获取到该序列,我们就可以通过递归的方式,精确地定位到数据包的发送源。
实验评估表明,本文提出的方案在溯源速度和准确性上均优于传统的PPM方案,能够有效地应用于网络溯源任务。\par
(2) 针对当前攻击检测模型无法充分利用流量数据中的空间特征与时序特征的问题,本文设计并提出了一种基于残差网络与双向门控循环单元多模态特征融合的模型。
该模型通过结合残差网络与双向门控循环单元的特性,有效挖掘流量数据的时空特征,并综合分析产生更准确的决策。
在UNSW-NB15数据集上的实验对比表明,该模型在准确率方面相较于传统机器学习模型以及ResNet、BiGRU等模型具有显著优势。\par
(3) 针对攻击检测模型难以适应不断变化的网络攻击环境的问题,本文结合提出的RNB-MF模型以及iCaRL策略,设计了一套增量学习方案。
该方案通过引入本文所提出的基于遗传算法的记忆集优化方法,进一步提升了方案的性能表现。
在UNSW-NB15和NSL-KDD数据集上的实验验证表明,该方案能够有效地支持模型的增量学习和知识保留,使模型不断适应日益变化的网络攻击环境。\par

\section{前景展望}
尽管我们在网络溯源和网络攻击检测方面取得了一定的研究成果,但当前的工作仍然存在一些局限性和待改进之处。以下是关于本文工作可以继续优化的几个方向:\par

(1)对于路由器接口号概率包标记方案,虽然本文已经提出了获取接口号序列路径的有效方法,但目前的方案仍然需要通过轮询的方式去恢复具体的物理路径。
为了进一步提高效率,未来的研究可以探索如何利用该接口号路径序列直接还原为实际物理路径,而无需查询路由器。
这将有助于减少网络溯源过程中的时间成本,提高溯源的速度和准确性。\par

(2)在评估RNM-MF模型时,我们采用了UNSW-NB15数据集。
然而,这个数据集存在严重的不平衡问题,导致某些攻击类型由于样本数量不足,其检测性能未能得到显著提升。
尽管我们采用了过采样技术来缓解这一问题,但如何进一步提高模型在处理类似不平衡数据集时对少数类的检测准确率仍然是一个值得深入研究的问题。
未来的研究可以考虑采用更先进的采样技术、损失函数优化方法或模型架构调整等手段,以提升模型在不平衡数据集上的性能表现。\par