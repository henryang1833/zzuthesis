\chapter{总结与展望}
\label{cha:over_view}

\section{本文总结}


随着移动互联网的蓬勃发展和人工智能技术的不断革新,互联网已深入渗透到人们生活的每个角落,每日产生的网络流量数据呈爆炸式增长。
这种数据量的激增在极大丰富人们生活的同时,也带来了日益严峻的安全挑战。
针对这些挑战,本文做出了如下工作:\par
1)~针对当前攻击检测模型无法充分利用流量数据中的空间特征与时序特征的问题,本文设计并提出了一种基于残差网络与双向门控循环单元多模态特征融合模型。
该模型通过结合残差网络与双向门控循环单元的特性,有效挖掘流量数据的时空特征,并综合分析产生更准确的决策。
在UNSW-NB15数据集上的实验结果显示,该模型准确率高达99.46\%,并在各类攻击样本的精度和召回率指标上也取得了卓越表现。\par

2)~针对攻击检测模型难以适应不断变化的网络攻击环境的问题,本文通过引入iCaRL策略,对模型进行了功能增强,使模型具备了增量学习机制。
为进一步提升模型的增量学习效果,本文又提出了基于遗传算法的记忆集优化方法。
通过利用UNSW-NB15和NSL-KDD数据集分别作为新旧任务数据集对模型进行实验验证,实验结果显示,该模型在新旧任务混合数据集上准确率高达78.1\%,在旧任务数据集上准确率达75.6\%。
结果表明,增强后的模型能够有效地进行增量学习和知识保留,从而不断适应日益变化的网络攻击环境。\par


3)~在实际运行中,检测模型常常面临DDoS攻击的威胁,特别是攻击者不断伪造数据包IP地址的情况。
在这种情况下,服务器往往难以有效过滤攻击流量。
一旦无法及时过滤这些攻击流量,检测模型将面临巨大的负载压力,甚至可能因过载而崩溃。
针对此问题,本文提出了基于数据包标记溯源的异常流量检测模型防护方法。
该方法的核心在于本文创新的路由器接口号成组标记溯源方法。
为验证该溯源方法的效果,本文首先利用CAIDA IPv4 Prefix-Probing Traceroute数据集对其进行了溯源速度测试。
实验结果表明,在使用“组数”为4的条件下,仅需收集65个数据包,该方法便能成功追踪到距离接受结点30跳路由器的数据发送结点。
紧接着,我们进一步利用NS-3模拟器构造了一个网络仿真环境,以测试该方法在面对实际DDoS攻击时的溯源准确率。
实验结果显示,随着攻击节点数量不断增加,直至达到100个的情况下,溯源准确率依然能够稳定在90\%以上。
最后,我们对整体防护方法进行了详细阐述,并通过VMWare WorkStation构建了一个高度仿真的网络攻击环境,对防护方法进行了实验验证。
实验结果表明,该防护方法能够有效定位攻击者的最近邻路由器,并依托该路由器实现对攻击流量的过滤,从而保障检测模型的稳定运行和服务器的安全无虞。

\section{前景展望}
尽管我们在网络溯源和网络攻击检测方面取得了一定的研究成果,但当前的工作仍然存在一些局限性和待改进之处。以下是关于本文工作可以继续优化的几个方向:\par

1)~对于路由器接口号成组标记溯源方法,虽然本文已经提出了获取接口号序列路径的有效方法,但目前的方案仍然需要通过轮询的方式去恢复具体的物理路径。
为了进一步提高效率,未来的研究可以探索如何利用该接口号路径序列直接还原为实际物理路径,而无需查询路由器。
这将有助于减少网络溯源过程中的时间成本,提高溯源的速度和准确性。\par

2)~在评估RNM-MF模型时,我们采用了UNSW-NB15数据集。
然而,这个数据集存在严重的不平衡问题,导致某些攻击类型由于样本数量不足,其检测性能未能得到显著提升。
尽管我们采用了过采样技术来缓解这一问题,但如何进一步提高模型在处理类似不平衡数据集时对少数类的检测准确率仍然是一个值得深入研究的问题。
未来的研究可以考虑采用更先进的采样技术、损失函数优化方法或模型架构调整等手段,以提升模型在不平衡数据集上的性能表现。\par